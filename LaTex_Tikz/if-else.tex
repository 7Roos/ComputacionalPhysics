\documentclass[12pt]{article}
%\usepackage{tikz}
\usepackage[T1]{fontenc}
\usepackage[utf8]{inputenc}
\usepackage[margin=2cm]{geometry}
\usepackage{xcolor}
\usepackage{fancyhdr}
\usepackage{longtable} % Enable break rows of the table

\usepackage{ifthen}
%\usepackage{etoolbox}

\usepackage{listings}
\lstset{
	language=[LaTeX]TeX,
	basicstyle=\ttfamily,
	keepspaces=true,
	columns=fixed,
	breaklines=true,
	commentstyle=\color{green!50!black},
	keywordstyle=\color{blue!70!red},
	stringstyle=\color{red},
	backgroundcolor=\color{yellow!40},
	frame=single,
	framesep=0pt,
	framerule=0pt,
	frameshape={RYR}{Y}{Y}{RYR},
	rulecolor=\color{gray!50},
	literate=
		{<i>}{{{\color{red}<i>}}}3
		{<j>}{{{\color{red}<j>}}}3
		{<x>}{{{\color{red}<x>}}}3
		{<y>}{{{\color{red}<y>}}}3
		{<Op>}{{{\color{teal!80!black}<Op>}}}4
		{\\ifboolFlag}{{\\color{violet}\\ifboolFlag}}1,
		moretexcs={ifboolFlag},
		texcsstyle=*\color{violet}
}

\renewcommand{\i}{1}
\renewcommand{\j}{2}
\renewcommand{\k}{-1}
\newcommand{\x}{1.1cm}
\newcommand{\y}{2pt}
\newcommand{\z}{2.0pt}
\newif\ifboolFlag \boolFlagtrue % set the boolean variable to true

% Add footer with variable values
\newcommand{\variablefooter}{
	\noindent\rule{\textwidth}{0.5pt}
	\footnotesize
	\begin{tabular}{l@{\hspace{1cm}}l@{\hspace{1cm}}l@{\hspace{1cm}}l@{\hspace{1cm}}l@{\hspace{1cm}}l@{\hspace{1cm}}l}
	  \textcolor{red}{$i = \i$} & 
	  \textcolor{blue}{$j = \j$} & 
	  \textcolor{green!60!black}{$k = \k$} & 
	  \textcolor{orange}{$x = \x$} & 
	  \textcolor{purple}{$y = \y$} & 
	  \textcolor{teal}{$z = \z$} & 
	  \textcolor{magenta}{\texttt{boolFlagtrue}} \\
	\end{tabular}
  }

  % Add the footer to each page
  \pagestyle{fancy}
  \fancyhf{}
  \renewcommand{\footrule}{\rule{\textwidth}{0pt}}
  \fancyfoot[C]{\variablefooter}

\title{Conditional Commands in \LaTeX}
\author{M. Roos}

\begin{document}
\maketitle
	
\section{Define parameter}
\begin{center}
		% These @{} directives remove the default padding space at the outer edges of the table, creating a more compact layout that aligns flush with the surrounding text.
		% Within this table, six mathematical expressions are displayed, each showing a variable name followed by its current value.
		%the tilde symbol (~) serves as a special character that creates a non-breaking space.
		\begin{tabular}{@{}cccccc@{}}
			$\bullet~i = \i$ & \hspace{0.5cm} $\bullet~j = \j$ & \hspace{0.5cm} $\bullet~k = \k$ & \hspace{0.5cm} $\bullet~x = \x$ & \hspace{0.5cm} $\bullet~y = \y$ & \hspace{0.5cm} $\bullet~z = \z$ \\
		\end{tabular}
\end{center}

\section{Native commands}
	
\subsection{The \texttt{\textbackslash ifodd} command}
	\noindent\textbf{Description:} Checks if a number \textcolor{violet}{\texttt{<i>}} is odd or even.

	\noindent\textbf{Inline syntax:}
	\begin{lstlisting}
\ifodd <i> <code if true> \else <code if false> \fi
	\end{lstlisting}

	\noindent\textbf{Block syntax:}
	\begin{lstlisting}
\ifodd <i>
<code executed if condition is true>
\else
<code executed if condition is false>
\fi
	\end{lstlisting}
	\begin{center}
	\begin{tabular}{l|c}
		\hline
		\textbf{Command} & \textbf{Result} \\
		\hline
		\verb|\ifodd| \texttt{\textbackslash i} & 
		\ifodd \i	{odd}	\else	{even}  \fi \\
		\verb|\ifodd| \texttt{\textbackslash j} & 
		\ifodd \j	{odd}	\else	{even}  \fi \\
		\verb|\ifodd| \texttt{\textbackslash k} & 
		\ifodd \k	{odd}	\else	{even}  \fi \\
		\verb|\ifodd| \texttt{\textbackslash y} & 
		\ifodd \y	{odd}	\else	{even}  \fi \\
		\verb|\ifodd| \texttt{\textbackslash z} & 
		\ifodd \z	{odd}	\else	{even}  \fi \\
		\hline
	\end{tabular}
	\end{center}
	\noindent\textbf{Example:} \verb|\ifodd \i {Odd} \else {Even} \fi| $\Longrightarrow$ produces "\ifodd \i Odd \else Even \fi".

%=================================================================
%=================================================================	

	\subsection{The \texttt{\textbackslash ifnum} command}
		\noindent\textbf{Description:} Compares two integer numbers \textcolor{violet}{\texttt{<i>}} and \textcolor{violet}{\texttt{<j>}} and conditionally executes code.

		\noindent\textbf{Valid Op:} \texttt{<} (less than), \texttt{=} (equal to), \texttt{>} (greater than)

		\noindent\textbf{Inline syntax:}
		\begin{lstlisting}
\ifnum <i> <Op> <j> <code if true> \else <code if false> \fi
		\end{lstlisting}

		\noindent\textbf{Block syntax:}
		\begin{lstlisting}
\ifnum <i> <Op> <j>
    <code executed if condition is true>
\else
    <code executed if condition is false>
\fi
		\end{lstlisting}

		\begin{center}
		\begin{tabular}{c|l|c}
			\hline
			\emph{Op} & \textbf{Command} & \textbf{Result} \\
			\hline
			\verb|<| & \verb|\ifnum| \texttt{\textbackslash i} \verb|<| \texttt{\textbackslash j} & 
			\ifnum \i < \j	{true}	\else	{false}  \fi \\
			\verb|<| & \verb|\ifnum| \texttt{\textbackslash i} \verb|<| \texttt{\textbackslash k} & 
			\ifnum \i < \k	{true}	\else	{false}  \fi \\
			\hline
			\verb|>| & \verb|\ifnum| \texttt{\textbackslash i} \verb|>| \texttt{\textbackslash j} & 
			\ifnum \i > \j	{true}	\else	{false}  \fi \\
			\verb|>| & \verb|\ifnum| \texttt{\textbackslash i} \verb|>| \texttt{\textbackslash k} & 
			\ifnum \i > \k	{true}	\else	{false}  \fi \\
			\hline
			\verb|=| & \verb|\ifnum| \texttt{\textbackslash i} \verb|=| \texttt{\textbackslash j} & 
			\ifnum \i = \j	{true}	\else	{false}  \fi \\
			\verb|=| & \verb|\ifnum| \texttt{\textbackslash i} \verb|=| \texttt{\textbackslash k} & 
			\ifnum \i = \k	{true}	\else	{false}  \fi \\
			\hline
		\end{tabular}
	\end{center}

	\noindent\textbf{Example:} 
		\begin{lstlisting}
\ifnum \i < \j 
   True
\else
   False
\fi %produces "\ifnum \i<\j True \else False \fi" 
		\end{lstlisting}

%=================================================================
%=================================================================	

	\subsection{The \texttt{\textbackslash ifdim} command}
		\noindent\textbf{Description:} Compares two dimensions \textcolor{violet}{\texttt{<x>}} and \textcolor{violet}{\texttt{<y>}} and conditionally executes code.

		\noindent\textbf{Valid Op:} \texttt{<} (less than), \texttt{=} (equal to), \texttt{>} (greater than)

		\noindent\textbf{Inline syntax:}
		\begin{lstlisting}
\ifdim <x> <Op> <y> <code if true> \else <code if false> \fi
		\end{lstlisting}
		
		\noindent\textbf{Block syntax:}
		\begin{lstlisting}[basicstyle=\ttfamily\small]
\ifdim <x> <Op> <y>
   <code executed if condition is true>
\else
   <code executed if condition is false>
\fi
		\end{lstlisting}
		
	\begin{center}
		\begin{tabular}{c|l|c}
			\hline
			\emph{Op} & \textbf{Command} & \textbf{Result} \\
			\hline
			\verb|<| & \verb|\ifdim| \texttt{\textbackslash x} \verb|<| \texttt{\textbackslash y} & 
			\ifdim \x < \y	{true}	\else	{false}  \fi \\
			\verb|<| & \verb|\ifdim| \texttt{\textbackslash z} \verb|<| \texttt{\textbackslash y} & 
			\ifdim \z < \y	{true}	\else	{false}  \fi \\
			\hline
			\verb|>| & \verb|\ifdim| \texttt{\textbackslash x} \verb|>| \texttt{\textbackslash y} & 
			\ifdim \x > \y	{true}	\else	{false}  \fi \\
			\verb|>| & \verb|\ifdim| \texttt{\textbackslash z} \verb|>| \texttt{\textbackslash y} & 
			\ifdim \z > \y	{true}	\else	{false}  \fi \\
			\hline
			\verb|=| & \verb|\ifdim| \texttt{\textbackslash x} \verb|=| \texttt{\textbackslash y} & 
			\ifdim \x = \y	{true}	\else	{false}  \fi \\
			\verb|=| & \verb|\ifdim| \texttt{\textbackslash z} \verb|=| \texttt{\textbackslash y} & 
			\ifdim \z = \y	{true}	\else	{false}  \fi \\
			\hline
		\end{tabular}
	\end{center}

	\noindent\textbf{Example:} \verb|\ifdim \x < \y True \else False \fi| $\Longrightarrow$ produces "\ifdim \x<\y True \else False \fi".

%=================================================================
%=================================================================		

	\subsection{The \texttt{\textbackslash ifcase} command}
		\noindent\textbf{Description:} Compares a number \textcolor{violet}{\texttt{<i>}} with a list of numbers and conditionally executes code. Implements a multiple selection structure, similar to the switch-case of other programming languages.

		\noindent\textbf{Inline syntax:}
		\begin{lstlisting}
\ifcase <i> <cond 0> \or <cond 1> \or <cond 2> \else <cond n> \fi
		\end{lstlisting}

		\noindent\textbf{Block syntax:}
		\begin{lstlisting}
\ifcase <i>
   <code executed if <i> is 0>
\or
   <code executed if <i> is 1>
\or
   <code executed if <i> is 2>
\else
   <code executed if <i> is n> % Standart case
\fi
		\end{lstlisting}

	\begin{center}
		\begin{tabular}{l|c}
			\hline
			\textbf{Command} & \textbf{Result} \\
			\hline
			\verb|\ifcase| \texttt{\textbackslash i} & 
			\ifcase \i	{condition 0} \or {condition 1}  \fi \\
			\verb|\ifcase| \texttt{\textbackslash j} & 
			\ifcase \j	{condition 0} \or {condition 1} \or {condition 2}  \fi \\
			\verb|\ifcase| \texttt{\textbackslash x} & 
			\ifcase \y	{condition 0} \or {condition 1} \or {condition 2}  \fi \\
			\verb|\ifcase| \texttt{\textbackslash y} & 
			\ifcase \z	{condition 0} \or {condition 1} \or {condition 2}  \fi \\
			\hline
		\end{tabular}
	\end{center}

	\noindent\textbf{Example:} \verb|\ifcase \i	{condition 0} \or {condition 1}  \fi| $\Longrightarrow$ produces "\ifcase \i {condition 0} \or {condition 1}  \fi".
	
%=================================================================
%=================================================================	

\subsection{The \texttt{\textbackslash ifx} or \texttt{\textbackslash if} command}

		\noindent\textbf{Description:} Compares two tokens \textcolor{violet}{\texttt{<token 1>}} and \textcolor{violet}{\texttt{<token 2>}} and conditionally executes code.

		\noindent\textbf{Inline syntax:}
		\begin{lstlisting}
\ifx <token 1> <token 2> <code if true> \else <code if false> \fi 
%or 
\if <token 1> <token 2> <code if true> \else <code if false> \fi 
		\end{lstlisting}

		\noindent\textbf{Block syntax:}
		\begin{lstlisting}
\ifx <token 1> <token 2>
   <code executed if token 1 is identical to token 2>
\else
   <code executed if condition is false>
\fi 
%or
\if <token 1> <token 2>
   <code executed if token 1 is identical to token 2>
\else
   <code executed if condition is false>
\fi
		\end{lstlisting}

	\begin{center}
		\begin{tabular}{l|c}
			\hline
			\textbf{Command} & \textbf{Result} \\
			\hline
			\verb|\ifx| \texttt{\textbackslash i} \texttt{\textbackslash j} & 
			\ifx \i \j	{true}	\else	{false}  \fi \\
			\verb|\ifx| \texttt{\textbackslash i} \texttt{\textbackslash k} & 
			\ifx \i \k	{true}	\else	{false}  \fi \\
			\verb|\ifx| \texttt{\textbackslash x} \texttt{\textbackslash y} & 
			\ifx \x \y	{true}	\else	{false}  \fi \\
			\verb|\ifx| \texttt{\textbackslash z} \texttt{\textbackslash y} & 
			\ifx \z \y	{true}	\else	{false}  \fi \\
			\hline
		\end{tabular}
	\end{center}

	\noindent\textbf{Example:} \verb|\ifx \i \j {True} \else {False} \fi| $\Longrightarrow$ produces "\ifx \i \j True \else False \fi".

%=================================================================
%=================================================================

	\subsection{The \texttt{\textbackslash newif} command}
	\noindent\textbf{Description:} You can create new conditionals (as a kind of boolean variables) with the \verb|\newif| command. With this self defined conditionals you can control the output of your code in an elegant way. Two versions of a document must be generated.

	\noindent\textbf{Inline syntax:}
	\begin{lstlisting}
\newif\if<boolFlag> \boolFlagtrue \ifboolFlag <code if true> \else <code if false> \fi
	\end{lstlisting}

	\noindent\textbf{Block syntax:}
	\begin{lstlisting}
\newif\if<boolFlag> % create a new boolean variable
\boolFlagtrue       % set the boolean variable to true
%\boolFlagfalse     % set the boolean variable to false
\ifboolFlag         % check the value of the boolean variable
   <code executed if condition is true>
\else
   <code executed if condition is false>
\fi
	\end{lstlisting}

	\begin{center}
		\begin{tabular}{l|c}
			\hline
			\textbf{Command} & \textbf{Result} \\
			\hline
			\verb|\ifboolFlag| & \ifboolFlag \textbf{True} \else \textbf{False} \fi \\ 
			\hline
		\end{tabular}
	\end{center}

	\noindent\textbf{Example:} 
	\begin{lstlisting}
\newif\ifOp % create a new boolean variable
\Optrue     % set the boolean variable to true
\ifOp       % check the value of the boolean variable
   \textbf{True}
\else
   \textbf{False}
\fi
	\end{lstlisting}
	\noindent produces "\newif\ifOp \Optrue \Opfalse \ifOp <code executed if condition is true> \else <code executed if condition is false> \fi".
%=================================================================
%=================================================================
	\subsection{The \texttt{\textbackslash ifdefined} command}
	\noindent\textbf{Description:} Checks if a command \textcolor{green!50!black}{\texttt{<Op>}} is defined or not.

	\noindent\textbf{Inline syntax:}
	\begin{lstlisting}
\ifdefined <Op> <code if true> \else <code if false> \fi
	\end{lstlisting}

	\noindent\textbf{Block syntax:}
	\begin{lstlisting}
\ifdefined <Op>
   <code executed if condition is true>
\else
   <code executed if condition is false>
\fi
	\end{lstlisting}

	\begin{center}
		\begin{tabular}{l|c}
			\hline
			\textbf{Command} & \textbf{Result} \\
			\hline
			\verb|\ifdefined| \texttt{\textbackslash i} & 
			\ifdefined \i	{defined}	\else	{undefined}  \fi \\
			\verb|\ifdefined| \texttt{\textbackslash j} & 
			\ifdefined \j	{defined}	\else	{undefined}  \fi \\
			\verb|\ifdefined| \texttt{\textbackslash k} & 
			\ifdefined \k	{defined}	\else	{undefined}  \fi \\
			\verb|\ifdefined| \texttt{\textbackslash abc} & 
			\ifdefined \abc	{defined}	\else	{undefined}  \fi \\
			\hline
		\end{tabular}
	\end{center}
	\noindent\textbf{Example:} \verb|\ifdefined \i {True} \else {False} \fi| $\Longrightarrow$ produces "\ifdefined \i True \else False \fi".
%=================================================================
%=================================================================

	\section{Packages}

	\subsection{The \texttt{\textbackslash ifthen} package}
	\noindent\textbf{Description:} Provides a more flexible and user-friendly way to handle conditional statements in \LaTeX. It allows use combinational logic operators \textcolor{red}{\texttt{and}}, \textcolor{red}{\texttt{or}} and \textcolor{red}{\texttt{not}}. Indeed, the pair of parenthesis \textcolor{red}{\texttt{()}} is used to group the boolean expressions. 
	However, the etoolbox package is recommended for new documents, because it is more modern and has a more consistent syntax. Indeed, the etoolbox package is developed with eLatex in mind, while the ifthen package is a legacy package.

	\subsubsection{The \texttt{\textbackslash ifthenelse} command}

		\noindent\textbf{Description:} Compares an integer positive \textcolor{violet}{\texttt{<i>}} and conditionally executes code.

		\noindent\textbf{Valid Op:} 
		\begin{itemize}
			\item Combinational logical operators: \texttt{and} and \texttt{or};
			\item Negation operator: \texttt{not};
			\item Relational logical operators: \texttt{<} (less than), \texttt{>} (greater than), \texttt{=} (equal to);
			\item \texttt{()} is used to group the boolean expressions;
			\item \verb|\isodd| check if a number is odd or even. Similar to \verb|\ifodd|;
			\item \verb|\isundefined| is true if \verb|\cmd| is not defined. Similar to \verb|\ifdefined|;
			\item \verb|\equal| check if two strings is equal or not. Similar to \verb|\ifx|;
			\item \verb|\lengthtest| check if a length is lesser, greater or equal to another
			 length. Similar to \verb|\ifdim|;
			\item \verb|\boolean| check if a boolean variable is true or false. Similar to \verb|\newif| mechanism.
		\end{itemize}

		\noindent\textbf{Inline syntax:}
		\begin{lstlisting}
\usepackage{ifthen}
\ifthenelse{<boolean expression>}{<code if true>}{<code if false>}
		\end{lstlisting}

		\noindent\textbf{Block syntax:}
		\begin{lstlisting}
\usepackage{ifthen}
\ifthenelse{<boolean expression>}
   {<code if true>
}{
   <code if false>}
		\end{lstlisting}

		\begin{center}
			\begin{longtable}{c|l|c}
				\hline
				\emph{Op} & \textbf{Command} & \textbf{Result} \\
				\hline
				\verb|<| & \verb|\ifthenelse{\i < \j}| & \ifthenelse{\i<\j}{true}{false} \\
				\verb|<| & \verb|\ifthenelse{\i < \k}| & \ifthenelse{\i<\k}{true}{false (\emph{Mathematical absurdity})} \\
				\verb|<| & \verb|\ifthenelse{\i < \x}| & \ifthenelse{\i<\x}{true}{false (\emph{Mathematical absurdity})} \\
				\verb|<| & \verb|\ifthenelse{\i < \y}| & \ifthenelse{\i<\y}{true (\emph{Undefined})}{false} \\
				\hline
				\verb|=| & \verb|\ifthenelse{\i = \j}| & \ifthenelse{\i=\j}{true}{false} \\
				\verb|=| & \verb|\ifthenelse{\i = \k}| & \ifthenelse{\i=\k}{true}{false} \\
				\verb|=| & \verb|\ifthenelse{\y = \z}| & \ifdefined \ifthenelse{\y=\z}{true}{false} {(\emph{Undefined})} \fi \\
				\hline
				\verb|\and| & \verb|\ifthenelse{\i < \j \and 0 < \i}| & \ifthenelse{\i<\j \and 0<\i}{true}{false} \\
				\hline
				\verb|\or| & \verb|\ifthenelse{\i < \j \or \i > \j}| & \ifthenelse{\i<\j \or \i>\j}{true}{false} \\
				\hline
				\verb|\not| & \verb|\ifthenelse{\not \i < \j}| & \ifthenelse{\not \i < \j}{true}{false} \\
				\hline
				\verb|\AND| & \verb|\ifthenelse{\i < \j \AND 0 < \i}| & \ifthenelse{\i < \j \AND 0 < \i}{true}{false} \\
				\hline
				\verb|\OR| & \verb|\ifthenelse{\i < \j \OR \i > \j}| & \ifthenelse{\i < \j \OR \i > \j}{true}{false} \\
				\hline
				\verb|\NOT| & \verb|\ifthenelse{\NOT \i < \j}| & \ifthenelse{\NOT \i < \j}{true}{false} \\ 
				\hline
				\verb|\(\)| & \verb|\ifthenelse{\(\i < \j\)}| & \ifthenelse{\(\i < \j\)}{true}{false} \\
				\hline
				\verb|\isodd| & \verb|\ifthenelse{\isodd{\i}}| & \ifthenelse{\isodd{\i}}{true}{false} \\
				\verb|\isodd| & \verb|\ifthenelse{\isodd{\j}}| & \ifthenelse{\isodd{\j}}{true}{false} \\
				\verb|\isodd| & \verb|\ifthenelse{\isodd{\k}}| & \ifthenelse{\isodd{\k}}{true}{false} \\
				\verb|\isodd| & \verb|\ifthenelse{\isodd{\y}}| & \ifthenelse{\isodd{\y}}{true}{false} \\
				\verb|\isodd| & \verb|\ifthenelse{\isodd{\z}}| & \ifthenelse{\isodd{\z}}{true}{false} \\
				\hline
				\verb|\isundefined| & \verb|\ifthenelse{\isundefined{\i}}| & \ifthenelse{\isundefined{\i}}{true}{false} \\
				\verb|\isundefined| & \verb|\ifthenelse{\isundefined{\j}}| & \ifthenelse{\isundefined{\j}}{true}{false} \\
				\verb|\isundefined| & \verb|\ifthenelse{\isundefined{\k}}| & \ifthenelse{\isundefined{ki}}{true}{false} \\
				\verb|\isundefined| & \verb|\ifthenelse{\isundefined{\abc}}| & \ifthenelse{\isundefined{\abc}}{true}{false} \\
				\hline
				\verb|\equal| & \verb|\ifthenelse{\equal{\i}{\i}}| & \ifthenelse{\equal{\i}{\i}}{true}{false} \\
				\verb|\equal| & \verb|\ifthenelse{\equal{\i}{1}}| & \ifthenelse{\equal{\i}{1}}{true}{false} \\
				\verb|\equal| & \verb|\ifthenelse{\equal{\i}{\j}}| & \ifthenelse{\equal{\i}{\j}}{true}{false} \\
				\verb|\equal| & \verb|\ifthenelse{\equal{\i}{\k}}| & \ifthenelse{\equal{\i}{\k}}{true}{false} \\
				\verb|\equal| & \verb|\ifthenelse{\equal{\x}{\y}}| & \ifthenelse{\equal{\x}{\y}}{true}{false} \\
				\verb|\equal| & \verb|\ifthenelse{\equal{\z}{\y}}| & \ifthenelse{\equal{\z}{\y}}{true}{false} \\
				\hline
				\verb|\lengthtest| and \verb|<| & \verb|\ifthenelse{\lengthtest{\x < \y}}| & \ifthenelse{\lengthtest{\x < \y}}{true}{false} \\
				\verb|\lengthtest| and \verb|<| & \verb|\ifthenelse{\lengthtest{\z < \y}}| & \ifthenelse{\lengthtest{\z < \y}}{true}{false} \\
				\hline
				\verb|\lengthtest| and \verb|>| & \verb|\ifthenelse{\lengthtest{\x > \y}}| & \ifthenelse{\lengthtest{\x > \y}}{true}{false} \\
				\verb|\lengthtest| and \verb|>| & \verb|\ifthenelse{\lengthtest{\z > \y}}| & \ifthenelse{\lengthtest{\z > \y}}{true}{false} \\
				\hline
				\verb|\lengthtest| and \verb|=| & \verb|\ifthenelse{\lengthtest{\x = \y}}| & \ifthenelse{\lengthtest{\x = \y}}{true}{false} \\
				\verb|\lengthtest| and \verb|=| & \verb|\ifthenelse{\lengthtest{\z = \y}}| & \ifthenelse{\lengthtest{\z = \y}}{true}{false} \\
				\hline
				\verb|\boolean| & \verb|\ifthenelse{\boolean{boolFlag}}| & \ifthenelse{\boolean{boolFlag}}{true}{false} \\
				\hline
			\end{longtable}
		\end{center}
	
		\noindent\textbf{Example:} \verb|\ifthenelse{\i<\j}{True}{False}| $\Longrightarrow$ produces "\ifthenelse{\i<\j}{True}{False}".

		\noindent\textbf{Extra:}
		\begin{itemize}
			\item \verb|\newboolean|: Creates a new boolean variable. If it the variable already exists, an error is thrown. Ex: \verb|\newboolean{boolFlag}|;
			\item \verb|\provideboolean|: Creates a new boolean variable, even if it already exists. 
			
			Ex: \verb|\provideboolean{boolFlag}|;
			\item \verb|\setboolean|: Sets the value of a boolean variable. Ex: \verb|\setboolean{boolFlag}{true}| or \verb|\setboolean{boolFlag}{false}|.
		\end{itemize}
%=================================================================
%=================================================================

	\subsection{The \texttt{\textbackslash etoolbox} package}
	\noindent\textbf{Description:} Provides a more flexible and user-friendly way to handle conditional statements in \LaTeX. This package provides two interfaces to boolean flags which are completely independent of each other. The tools in section 3.5.1 are a LaTeX frontend to \verb|\newif|. Those in section 3.5.2 use a different mechanism.

	\noindent\textbf{Inline syntax:}

	\begin{lstlisting}
\usepackage{etoolbox}
\ifboolexpr{<condition>}{<code if true>}{<code if false>}
	\end{lstlisting}
	\noindent\textbf{Block syntax:}
	\begin{lstlisting}
\usepackage{etoolbox}
\ifboolexpr{<condition>}
	{<code if true>
}{
	<code if false>}
	\end{lstlisting}
	\begin{center}
		\begin{tabular}{l|c}
			\hline
			\textbf{Command} & \textbf{Result} \\
			\hline
			\verb|\ifboolexpr{\i<\j}| &  \\ %\ifboolexpr{\i<\j}{true}{false}
			%\verb|\ifboolexpr{\i<\k}| & \ifboolexpr{\i<\k}{true}{false (\emph{Mathematical absurdity})} \\
			%\verb|\ifboolexpr{\i=\j}| & \ifboolexpr{\i=\j}{true}{false} \\
			%\verb|\ifboolexpr{\i=\k}| & \ifboolexpr{\i=\k}{true}{false} \\
			%\verb|\ifboolexpr{\i<\x}| & \ifboolexpr{\i<\x}{true}{false (\emph{Mathematical absurdity})} \\
			%\verb|\ifboolexpr{\i<\y}| & \ifboolexpr{\i<\y}{true (\emph{Undefined})}			{false} \\
			\hline
		\end{tabular}
	\end{center}
	%\noindent\textbf{Example:} \verb|\ifboolexpr{\i<\j}{True}{False}| $\Longrightarrow$ produces "\ifboolexpr{\i<\j}{True}{False}".

%=================================================================
%=================================================================

\subsection{The \texttt{\textbackslash xstring} package}

\end{document}